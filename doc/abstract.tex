% -*-LaTeX-*-

\begin{abstract}

Over the past decade, the ``Cloud'' has introduced successive major
innovations in the way developpers deploy and maintain their
applications: First, it started by relieving them from the
administration of physical machines (EC2~\cite{amazon:ec2},
Rackspace~\cite{rackspace:cloudservers},
Joyent~\cite{joyent:smartmachines}). Then, more recently it has freed
them from administration altogether by handling automatic deployment,
maintenance and scaling of apps on hosted clusters (App
Engine~\cite{google:appengine}, Heroku~\cite{heroku:cloudapp},
Nodejitsu~\cite{nodejitsu:nodejscloud}).

We argue that the next natural simplification to be introduced by the
``Cloud'' is the ability to solely rely on client-side code for
application developement by taking advantage of the various
cloud-based services available today over the network, through
APIs. Nevertheless, such ``mash-ups'' are still uncommon and reserved
to simple experimental projects---Applications developpers still
heavily rely on server-side code for most of their advanced projects.

We demonstrate that the reason why it is still impractical for
developpers to free themselves entirely from the use of server-side
code in most of their project is authentication. More precisely, it is
the inability to forward authentication from identity providers
(Email/Password, Twitter Oauth, Facebook Connect, Google OAuth, ...)
to data providers and service APIs. Finally, we advocate a new
authentication model for foreign APIs and present ``cloudful'', a
protocol for that model, as well as a client library providing the
missing authentication forwarding infrastructure to eventually enable
the design of advanced applications solely out of client code.

\end{abstract}

