% -*-LaTeX-*-

\section{Current Limitations}

With the ``APIs'' as well as the cross-origin techniques being available,
it is interesting to ask why such programming model is not more widely
used today and generaly reserved to simple demonstrational projects more
commonly referred as ``web mashups''. 

One important reason why server-side code is still widely used is
authentication: most projects rely on ACL-based protection of
their consumer data. Let's consider an application which would store
such data on a remote service through an API implementing some form of
access-control lists. If the consumer is not asked to authentify directly against
that API, there is no way for client code alone to securely prove the API
the identity of the current consumer for the application. Or,
no consumer in her right mind would agree to authentify against the 2,
3 (10?) APIs leveraged by the application she's attempting to use.

Applications developpers know that, that's the reason why they
generally implement their own authentication system based on a native
identity (email & password) or an identity provider (Oauth service such
as Facebook, Google, Twitter, ...) and use their server code to
forward that identity to the APIs they use and enforce their security
model.

That task is extremely common across projects, it is also sensitive in
terms of security and, more importantly, it is standardizable by a protocol
between clients, identity providers and remote service providers. For
these reasons, we propose \emph{coudful}, 

a protocol extension to
OAuth for identity forwarding to remote services in the context of an
application as well as a library to abstract identity management 


